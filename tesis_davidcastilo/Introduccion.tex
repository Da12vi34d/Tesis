\chapter{Introducción}

Con el fin de modelar algún fenómeno de interés de la natraleza se han propuesto modelos estocásticos, como la selección de un portafolio de mínima varianza, o la segmentación de individuos mediante un análisis de discriminante, por ejemplo. Estos modelos estocásticos suponen que los datos tienen alguna estructura de correlación dada, como es el caso del portafolio de mínima varianza que presupone una estructura de correlación normal; otros modelos requieren la estimación de alguna ley de probabilida, donde muchas veces se utiliza la distribución normal (o una mezcla finita normal), por su practicidad.   

Dicho lo anterior, la presente tesis fue motivada por la siguiente pregunta ¿Qué hacer cuando hay evidencia de que un conjunto de datos no sigue una ley de probabilidad normal? Varios investigadores han dado respuesta a esta pregunta, mediante la distribución $T$, o la incorporación de un sesgo a la distribución normal (o a la distribución $T$), o mediante el uso de cópulas, o mediante el uso de la distribución hiperbólica generalizada, por ejemplo. De aquí que el tema central de esta tesis sea el cómo estimar una distribución hiperbólica (caso particular de la distribución hiperbólica generalizada).

El problema de cómo ajustar una distribución hiperbólica se realizará mediante distribuciones tipo mezcla normal en esperanza y varianza, ya que está distribución permite simplificar el problema, también para realizar el ajuste se utilizará un enfoque bayesiano, ya que se adecua de manera natural a las distribuciones tipo mezcla normal, y además permite explotar el conocimiento a priori de los datos para la obtención de los parámetros de la distribución a estimar.

La presente tesis se divide un $6$ capítulos. En el capítulo $1$ se introduce de manera breve el paradigma bayesiano de inferencia, el cual es necesario para el tipo de estimación que se hará. En el capítulo $2$ se expone la notación a usar, así como las distribuciones necesarias para la estimación; también se habla brevemente de algunas características de las distribuciones mencionadas. En el capítulo $3$ se introducen las distribuciones tipo mezcla normal, para así poder caracterízar a la distribución hiperbólica; también se habla brevemente de algunas características de estas distribuciones. En el capítulo $4$ se construyen la función de verosimilitud, y densidades marginales completas necesarias para la estimación. En el capítulo $5$ se hablará de los resultados y conclusiones. Y por último, en el capítulo $6$ se incluye como anexo algunos resultados utilizados utilizados en la presente tesis, así como los respectivos códigos implementados, es importante mencionar que para la obtención de resultados se utilizó el software $R$.


 